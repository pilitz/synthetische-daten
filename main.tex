% This is the main entry point for the thesis template. It includes all other
% smaller subfiles for each chapter and composes them to the complete document

% Each file starts with this command. It is used to define the corresponding
% base path of the template. This avoids the usage of absolute paths.

% The thesis template is defined in its own document class.
% Please see the zyx-directory for more details
\documentclass[oneside,de]{./zyx/mythesis}
%% If you prefer one side per page use this instead of the previous line
% \documentclass[oneside]{./zyx/mythesis}
%% If you insist on writing your thesis in german add this option
% \documentclass[twoside,de]{./zyx/mythesis}
%% If all the front pages are annoying you, try the following
% \documentclass[twoside,wip]{./zyx/mythesis}

% Add the library file for the literature references
\addbibresource{./literature.bib}

% Load acronyms file
\input{./acronyms.tex}

\begin{document}
% Cover, abstract, table of contents, etc.
\subfile{./zyx/preface.tex}

%% Add more chapters like this
\subfile{./txt/1-einleitung.tex}
\subfile{./txt/2-grundlagen.tex}
\subfile{./txt/3-erstellung-synthetischer-daten.tex}
\subfile{./txt/4-vorteile.tex}
\subfile{./txt/5-herausforderungen.tex}
\subfile{./txt/6-anwendungsfaelle.tex}
\subfile{./txt/7-zukunft.tex}
\subfile{./txt/8-fazit.tex}

% Bibliography, list of tables, list of figures, etc.
\subfile{./zyx/epilogue.tex}

% Add and appendix to the document
\cleardoublepage
\appendix
\subfile{./txt/99-lipsum.tex}

\end{document}
