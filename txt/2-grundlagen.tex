\documentclass[../main.tex]{subfiles}

\begin{document}
\chapter{Grundlagen}

Synthetische Daten haben in den letzten Jahren an Bedeutung gewonnen, insbesondere im Kontext von Datenschutz und Datennutzung. Sie bieten eine innovative Lösung, um sensible Informationen zu schützen und gleichzeitig die Analyse und Verarbeitung von Daten zu ermöglichen. Im Rahmen der folgenden Grundlagenkapitel werden zentrale Aspekte des Datenschutzes und die spezifischen Anforderungen an den Umgang mit Daten erläutert. Anschließend erfolgt eine Einführung in das Konzept der synthetischen Daten, einschließlich ihrer Definition, verschiedener Typen und ihrer Abgrenzung zu anonymisierten und pseudonymisierten Daten. Praxisnahe Beispiele verdeutlichen abschließend, wie synthetische Daten in realen Szenarien eingesetzt werden können. Dieses Kapitel bildet somit die Basis für ein vertieftes Verständnis der Rolle synthetischer Daten im Datenschutz und deren Potenzial, technische und rechtliche Herausforderungen zu adressieren.

\section{Datenschutz und Anforderungen}

Datenschutz ist ein grundlegender Bestandteil moderner Informationsverarbeitung und ein zentraler Aspekt für Unternehmen, Institutionen und Einzelpersonen. Mit der zunehmenden Digitalisierung und der wachsenden Bedeutung von Daten als wertvolle Ressource steigen auch die Anforderungen an deren Schutz. Dieses Kapitel beleuchtet die rechtlichen Grundlagen, die den Umgang mit personenbezogenen Daten regeln, sowie die technischen Herausforderungen, die bei der Umsetzung eines effektiven Datenschutzes entstehen. Ziel ist es, ein Verständnis für die Komplexität und Vielschichtigkeit der Anforderungen zu schaffen, die als Grundlage für den Einsatz innovativer Technologien wie synthetische Daten dienen.

\subsection{Rechtliche Grundlagen}

Der Schutz personenbezogener Daten ist in der Europäischen Union durch die Datenschutz-Grundverordnung (DSGVO) umfassend geregelt. Ziel der DSGVO ist es, die Privatsphäre von Einzelpersonen zu wahren und die Verarbeitung personenbezogener Daten transparent, sicher und datenschutzkonform zu gestalten. Zentrale Prinzipien wie Datenminimierung, Zweckbindung und Transparenz bilden das Fundament der gesetzlichen Anforderungen und setzen klare Grenzen für den Umgang mit sensiblen Informationen.

Ein zentraler Aspekt der DSGVO ist die Definition personenbezogener Daten, die sich auf alle Informationen bezieht, die eine natürliche Person direkt oder indirekt identifizierbar machen (Art. 4 Nr. 1 DSGVO). Daraus ergibt sich die Verpflichtung, Verfahren wie Anonymisierung oder Pseudonymisierung einzusetzen, um die Identifizierbarkeit von Personen zu reduzieren. Besonders sensible Daten, wie Gesundheitsdaten oder biometrische Informationen, unterliegen gemäß Art. 9 DSGVO zusätzlichen Restriktionen, was die Nutzung dieser Daten in Analysen oder Anwendungen erschwert.

Die Rechte der betroffenen Personen, wie etwa das Recht auf Auskunft (Art. 15 DSGVO), Löschung (Art. 17 DSGVO) oder Datenübertragbarkeit (Art. 20 DSGVO), stellen darüber hinaus spezifische Anforderungen an die Datenverarbeitung. Systeme und Prozesse müssen so gestaltet werden, dass diese Rechte gewährleistet und Anfragen effizient bearbeitet werden können.

In diesem Spannungsfeld zwischen strengen gesetzlichen Vorgaben und der wachsenden Nachfrage nach datengetriebenen Innovationen gewinnen synthetische Daten an Bedeutung. Sie bieten eine Möglichkeit, die gesetzlichen Anforderungen an den Datenschutz zu erfüllen, indem sie realitätsnahe, aber künstlich erzeugte Datensätze bereitstellen. Da synthetische Daten keine direkten Rückschlüsse auf Einzelpersonen zulassen, können sie unter Umständen als datenschutzkonform gelten und erlauben eine risikofreie Nutzung in Forschung, Entwicklung und Analysen. Ihre Fähigkeit, den Konflikt zwischen Datenschutz und Datenverfügbarkeit zu entschärfen, macht sie zu einer vielversprechenden Technologie im Zeitalter der DSGVO.

\subsection{Technische Herausforderungen}
\section{Was sind synthetische Daten?}
\subsection{Definition und Arten}
\subsubsection{Teilsynthetische Daten}
\subsubsection{Vollsynthetische Daten}
\subsubsection{Hybride Daten}
\subsection{Unterschied zu anonymisierten und pseudonymisierten Daten}
\subsection{Beispiele für synthetische Daten}
\end{document}
